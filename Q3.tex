%نام و نام خانوادگی:
%شماره دانشجویی: 
\مسئله{نام سؤال}


\پاسخ{}

این گرامر LR(1) است. برای اثبات، آتاماتا و جدول پارس را رسم می‌کنیم تا عدم وجود کانفلیکت را نشان دهیم.

گرامر را به این صورت بازنویسی می‌کنیم:

\begin{center}
\begin{latin}
0.$S \rightarrow E$
\\
1.$E \rightarrow id$
\\
2.$E \rightarrow id(E)$
\\
3.$E \rightarrow E+id$
\end{latin}
\end{center}

که با توجه به گرامر نوشته شده، می‌توان این آتاماتا را رسم کرد:

\begin{latin}
    \begin{tikzpicture}
        [>=latex, 
        node distance=2.5cm, 
        block/.style={state, rectangle, text width=8em}
        ]
        \node [block, label=above left: 0] (q0) 
        {
            \(S \rightarrow .E \hfill \$\)
            \\
            \(E \rightarrow .id \hfill \$\)
            \\
            \(E \rightarrow .id \hfill +\)
            \\
            \(E \rightarrow .id(E) \hfill \$\)
            \\
            \(E \rightarrow .id(E) \hfill +\)
            \\
            \(E \rightarrow .E+id \hfill \$\)
            \\
            \(E \rightarrow .E+id \hfill +\)
            \\
        };

        \node [block, label=above left: 1] (q1) [right=1cm of q0]
        {
            \(S \rightarrow E. \hfill \$\)
            \\
            \(E \rightarrow E.+id \hfill \$\)
            \\
            \(E \rightarrow E.+id \hfill +\)
            \\
        };

        \node [block, label=above left: 2] (q2) [below=1cm of q0]
        {
            \(E \rightarrow id. \hfill \$\)
            \\
            \(E \rightarrow id. \hfill +\)
            \\
            \(E \rightarrow id.(E) \hfill \$\)
            \\
            \(E \rightarrow id.(E) \hfill +\)
            \\
        };
    
        \draw[->] (q0) edge[right] node[above] {E} (q1);
        \draw[->] (q0) edge[below] node[right] {id} (q2);
    \end{tikzpicture} 
\end{latin}