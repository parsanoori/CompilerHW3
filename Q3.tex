%نام و نام خانوادگی:
%شماره دانشجویی: 
\مسئله{نام سؤال}


\پاسخ{}

این گرامر LR(1) است. برای اثبات، آتاماتا و جدول پارس را رسم می‌کنیم تا عدم وجود کانفلیکت را نشان دهیم.

گرامر را به این صورت بازنویسی می‌کنیم:

\begin{center}
\begin{latin}
0.$S \rightarrow E$
\\
1.$E \rightarrow id$
\\
2.$E \rightarrow id(E)$
\\
3.$E \rightarrow E+id$
\end{latin}
\end{center}

که با توجه به گرامر نوشته شده، می‌توان این آتاماتا را رسم کرد:

\hspace*{-2cm}
\begin{latin}
    \begin{tikzpicture}
        [>=latex, 
        node distance=2.5cm, 
        block/.style={state, rectangle, text width=8em}
        ]
        \node [block, label=above left: 0] (q0) 
        {
            \(S \rightarrow .E \hfill \$\)
            \\
            \(E \rightarrow .id \hfill \$\)
            \\
            \(E \rightarrow .id \hfill +\)
            \\
            \(E \rightarrow .id(E) \hfill \$\)
            \\
            \(E \rightarrow .id(E) \hfill +\)
            \\
            \(E \rightarrow .E+id \hfill \$\)
            \\
            \(E \rightarrow .E+id \hfill +\)
            \\
        };

        \node [block, label=above left: 1] (q1) [right=1cm of q0]
        {
            \(S \rightarrow E. \hfill \$\)
            \\
            \(E \rightarrow E.+id \hfill \$\)
            \\
            \(E \rightarrow E.+id \hfill +\)
            \\
        };

        \node [block, label=above left: 2] (q2) [below=1cm of q0]
        {
            \(E \rightarrow id. \hfill \$\)
            \\
            \(E \rightarrow id. \hfill +\)
            \\
            \(E \rightarrow id.(E) \hfill \$\)
            \\
            \(E \rightarrow id.(E) \hfill +\)
            \\
        };

        \node [block, label=above left: 3] (q3) [right=1cm of q1]
        {
            \(E \rightarrow E+.id \hfill \$\)
            \\
            \(E \rightarrow E+.id \hfill +\)
            \\
        };

        \node [block, label=above left: 4] (q4) [below=1cm of q2]
        {
            \(E \rightarrow id(.E) \hfill \$\)
            \\
            \(E \rightarrow id(.E) \hfill +\)
            \\
            \(E \rightarrow .id \hfill +\)
            \\
            \(E \rightarrow .id \hfill )\)
            \\
            \(E \rightarrow .id(E) \hfill +\)
            \\
            \(E \rightarrow .id(E) \hfill )\)
            \\
            \(E \rightarrow .E+id \hfill +\)
            \\
            \(E \rightarrow .E+id \hfill )\)
            \\
        };

        \node [block, label=above left: 5] (q5) [right=1cm of q3]
        {
            \(E \rightarrow E+id. \hfill \$\)
            \\
            \(E \rightarrow E+id. \hfill +\)
            \\
        };

        \node [block, label=above left: 6] (q6) [below=1cm of q4]
        {
            \(E \rightarrow id(E.) \hfill \$\)
            \\
            \(E \rightarrow id(E.) \hfill +\)
            \\
            \(E \rightarrow E.+id \hfill +\)
            \\
            \(E \rightarrow E.+id \hfill )\)
            \\
        };

        \node [block, label=above left: 7] (q7) [right=1cm of q4]
        {
            \(E \rightarrow id.(E) \hfill +\)
            \\
            \(E \rightarrow id.(E) \hfill )\)
            \\
            \(E \rightarrow id. \hfill +\)
            \\
            \(E \rightarrow id. \hfill )\)
            \\
        };

        \node [block, label=above left: 8] (q8) [right=1cm of q6]
        {
            \(E \rightarrow id(E). \hfill \$\)
            \\
            \(E \rightarrow id(E). \hfill +\)
            \\
        };

        \node [block, label=above left: 9] (q9) [below=1cm of q6]
        {
            \(E \rightarrow E+.id \hfill +\)
            \\
            \(E \rightarrow E+.id \hfill )\)
            \\
        };

        \node [block, label=above left: 10] (q10) [right=1cm of q7]
        {
            \(E \rightarrow id(.E) \hfill +\)
            \\
            \(E \rightarrow id(.E) \hfill )\)
            \\
            \(E \rightarrow .id \hfill +\)
            \\
            \(E \rightarrow .id \hfill )\)
            \\
            \(E \rightarrow .id(E) \hfill +\)
            \\
            \(E \rightarrow .id(E) \hfill )\)
            \\
            \(E \rightarrow .E+id \hfill +\)
            \\
            \(E \rightarrow .E+id \hfill )\)
            \\
        };
        
        \node [block, label=above left: 11] (q11) [right=1cm of q9]
        {
            \(E \rightarrow E+id. \hfill +\)
            \\
            \(E \rightarrow E+id. \hfill )\)
            \\
        };

        \node [block, label=above left: 12] (q12) [right=1cm of q10]
        {
            \(E \rightarrow id(E.) \hfill +\)
            \\
            \(E \rightarrow id(E.) \hfill )\)
            \\
            \(E \rightarrow E.+id \hfill +\)
            \\
            \(E \rightarrow E.+id \hfill )\)
            \\
        };

        \node [block, label=above left: 13] (q13) [above=1cm of q12]
        {
            \(E \rightarrow id(E). \hfill +\)
            \\
            \(E \rightarrow id(E). \hfill )\)
            \\
        };
    
        \draw[->] (q0) edge[right] node[above] {E} (q1);
        \draw[->] (q0) edge[below] node[right] {id} (q2);
        \draw[->] (q1) edge[right] node[above] {+} (q3);
        \draw[->] (q2) edge[below] node[right] {(} (q4);
        \draw[->] (q3) edge[right] node[above] {id} (q5);
        \draw[->] (q4) edge[below] node[right] {E} (q6);
        \draw[->] (q4) edge[right] node[above] {id} (q7);
        \draw[->] (q6) edge[right] node[above] {)} (q8);
        \draw[->] (q6) edge[below] node[right] {+} (q9);
        \draw[->] (q7) edge[right] node[above] {(} (q10);
        \draw[->] (q9) edge[right] node[above] {id} (q11);
        \draw[->] (q10) edge[right] node[above] {E} (q12);
        \draw[->] (q10) edge[bend left,looseness=0.5,out=320,in=255] node[above] {id} (q7);
        \draw[->] (q12) edge[bend left,looseness=1,in=100] node[above] {+} (q9);
        \draw[->] (q12) edge[above] node[right] {)} (q13);
    \end{tikzpicture} 
\end{latin}

در ادامه، جدول پارس این گرامر را رسم خواهیم کرد، این جدول بدون هیچ کانفلیکتی قابل رسم است، پس این گرامر LR(1) خواهد بود:

\begin{center}
    \begin{latin}
    \begin{tabular}{|c|c|c|c|c|c|c|c|}
    \hline
    State & id & ( & ) & + & \$ & S & E \\ \hline
    0     & s2 &   &   &   &    &   & g1\\ \hline
    1     &    &   &   & s3& acc&   &   \\ \hline
    2     &    & s4&   & r1& r1 &   &   \\ \hline
    3     & s5 &   &   &   &    &   &   \\ \hline
    4     & s7 &   &   &   &    &   & g6\\ \hline
    5     &    &   &   & r3& r3 &   &   \\ \hline
    6     &    &   & s8& s9&    &   &   \\ \hline
    7     &    &s10& r1& r1&    &   &   \\ \hline
    8     &    &   &   & r2& r2 &   &   \\ \hline
    9     & s11&   &   &   &    &   &   \\ \hline
    10    & s7 &   &   &   &    &   &g12\\ \hline
    11    & s2 &   & r3& r3&    &   &   \\ \hline
    12    & s2 &   &s13& s9&    &   &   \\ \hline
    13    & s2 &   & r2& r2&    &   &   \\ \hline
    \end{tabular}
    \end{latin}
\end{center}
