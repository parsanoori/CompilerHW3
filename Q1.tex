%نام و نام خانوادگی:
%شماره دانشجویی: 
\مسئله{نام سؤال}
\پاسخ{}

ابتدا اپسیلون را حذف می‌کنیم.

\begin{center}
$S \rightarrow P \; | \; \epsilon$
\\*
$P \rightarrow () \; | \; (P) \; | \; ()P \; | \; (P)P$
\end{center}
به یافتن first و follow می‌پردازیم:

\begin{center}
    \begin{latin}
        \begin{tabular}{|c|c|c|}
            \hline
            Non Terminal & first & follow \\ \hline
            S            & (     & \$     \\ \hline
            P            & (     & \$,)  \\ \hline
        \end{tabular}
    \end{latin}
\end{center}

سپس به ساخت DFA می‌پردازیم.

\begin{latin}
\begin{tikzpicture}
  [>=latex, 
    node distance=5cm, 
    block/.style={state, rectangle, text width=8em}
  ]

    \node [block, label=above right: 0] (q0) 
    {
        \( S \rightarrow . P \hfill \$\)
        \\
        \( P \rightarrow . () \hfill ),\$\)
        \\
        \( P \rightarrow . (P) \hfill ),\$\)
        \\
        \( P \rightarrow . ()P \hfill ),\$\)
        \\
        \( P \rightarrow . (P)P \hfill ),\$\)
    };

    \node [block, label=above right: 1] (q1) [above of=q0]
    {
        \( S \rightarrow P . \hfill \$\)
    };

    \node [block, label=above right: 2] (q2) [right of=q0] 
    {
        \( P \rightarrow (.) \hfill ),\$\)
        \\
        \( P \rightarrow (.P) \hfill ),\$\)
        \\
        \( P \rightarrow (.)P \hfill ),\$\)
        \\
        \( P \rightarrow (.P)P \hfill ),\$\)

        \( P \rightarrow . () \hfill ),\$\)
        \\
        \( P \rightarrow . (P) \hfill ),\$\)
        \\
        \( P \rightarrow . ()P \hfill ),\$\)
        \\
        \( P \rightarrow . (P)P \hfill ),\$\)
    };

    \node [block, label=above right: 3] (q3) [right of=q1] 
    {
        \( P \rightarrow (). \hfill ),\$\)
        \\
        \( P \rightarrow ().P \hfill ),\$\)

        \( P \rightarrow . () \hfill ),\$\)
        \\
        \( P \rightarrow . (P) \hfill ),\$\)
        \\
        \( P \rightarrow . ()P \hfill ),\$\)
        \\
        \( P \rightarrow . (P)P \hfill ),\$\)
    };

    \node [block, label=above right: 4] (q4) [right of=q3] 
    {
        \( P \rightarrow ()P. \hfill ),\$\)
    };

    \node [block, label=above left: 5] (q5) [below of=q2] 
    {
        \( P \rightarrow (P.) \hfill ),\$\)
        \( P \rightarrow (P.)P \hfill ),\$\)
    };

    \node [block, label=above right: 6] (q6) [right of=q2] 
    {
        \( P \rightarrow (P). \hfill ),\$\)
        \\
        \( P \rightarrow (P).P \hfill ),\$\)
        \\
        \( P \rightarrow . () \hfill ),\$\)
        \\
        \( P \rightarrow . (P) \hfill ),\$\)
        \\
        \( P \rightarrow . ()P \hfill ),\$\)
        \\
        \( P \rightarrow . (P)P \hfill ),\$\)
    };

    \node [block, label=above right: 7] (q7) [below of=q6] 
    {
        \( P \rightarrow (P)P. \hfill ),\$\)
    };


    \path[->] (q0) edge [above] node[right] {P} (q1);
    \path[->] (q0) edge [above] node[above] {(} (q2);
    \path[->] (q2) edge [bend left] node[left] {)} (q3);
    \path[->] (q3) edge [bend left] node[right] {(} (q2);
    \path[->] (q3) edge [above] node[above] {P} (q4);
    \path[->] (q2) edge [above] node[left] {P} (q5);
    \path[->] (q5) edge [above] node[above] {)} (q6);
    \path[->] (q6) edge [above] node[above] {(} (q2);
    \path[->] (q6) edge [above] node[right] {P} (q7);

    \node [yshift=1cm] (acc) at (q1.north) {accept};
    \draw[->] (q1) edge[above] node[left] {\$} (acc);

             
\end{tikzpicture} 
\end{latin}

جدول SLR(1) به صورت زیر است:

\begin{latin}
\begin{center}
\begin{tabular}{|c|c|c|c|c|c|}
\hline
State & (  & )                       & \$                      & S & P  \\ \hline
0     & S2 &                         &                         &   & G1 \\ \hline
1     &    &                         & accept                  &   &    \\ \hline
2     &    & S3                      &                         &   & G5 \\ \hline
3     & S2 & R(P-\textgreater{}())   & R(P-\textgreater{}())   &   & G4 \\ \hline
4     &    & R(P -\textgreater ()P)  & R(P -\textgreater ()P)  &   &    \\ \hline
5     &    & S6                      &                         &   &    \\ \hline
6     & S2 &                         &                         &   & 7  \\ \hline
7     &    & R(P -\textgreater (P)P) & R(P -\textgreater (P)P) &   &    \\ \hline
\end{tabular}
\end{center}
\end{latin}
