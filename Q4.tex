%نام و نام خانوادگی:
%شماره دانشجویی: 
\مسئله{نام سؤال}

\پاسخ{}

قسمت الف:

\begin{center}
    \begin{latin}
    \begin{tabular}{|c|c|c|c|c|c|c|c|c|}
    \hline
    State  & |        & *        & a        & (        & )        & \$        & S        & R \\ \hline
    0      & r5       & r5       &s2/r5     &s3/r5     &          &r5         &          & g1\\ \hline
    1      &s5/r5     &s6/r5     &s2/r5     &s3/r5     &          &acc/r5     &          & g4\\ \hline
    2      &r6        &r6        &r6        &r6        &          &r6         &          &   \\ \hline
    3      &r5        &r5        &s8/r5     &s9/r5     &r5        &           &          & g7\\ \hline
    4      &s5/r2/r5  &s6/r2/r5  &s2/r2/r5  &s3/r2/r5  &          &r2/r5      &          & g4\\ \hline
    5      &r5        &r5        &s2/r5     &s3/r5     &          &r5         &          &g10\\ \hline
    6      &r4        &r4        &r4        &r4        &          &r4         &          &   \\ \hline
    7      &s13/r5    &s14/r5    &s8/r5     &s9/r5     &s11/r5    &           &          &g12\\ \hline
    8      &r6        &r6        &r6        &r6        &r6        &           &          &   \\ \hline
    9      &r5        &r5        &s8/r5     &s9/r5     &r5        &           &          &g15\\ \hline
    10     &s5/r3/r5  &s6/r3/r5  &s2/r3/r5  &s3/r3/r5  &          &r3/r5      &          & g4\\ \hline
    11     &r7        &r7        &r7        &r7        &          &r7         &          &   \\ \hline
    12     &s13/r2/r5 &s14/r2/r5 &s8/r2/r5  &s9/r2/r5  &r2/r5     &           &          &g12\\ \hline
    13     &r5        &r5        &s8/r5     &s9/r5     &r5        &           &          &g16\\ \hline
    14     &r4        &r4        &r4        &r4        &r4        &           &          &   \\ \hline
    15     &s13/r5    &s14/r5    &s8/r5     &s9/r5     &s17/r5    &           &          &g12\\ \hline
    16     &s13/r3/r5 &s14/r3/r5 &s8/r3/r5  &s9/r3/r5  &r3/r5     &           &          &g12\\ \hline
    17     &r7        &r7        &r7        &r7        &r7        &           &          &   \\ \hline
    \end{tabular}
    \end{latin}
\end{center}

قسمت ب:

کانفلیکت‌ها در جدول مشخص هستند، هر خانه که بیش از یک امکان برای آن وجود دارد، یک کانفلیکت خواهد بود.
هر گرامری که SLR(1) است حتما LR(1) هم خواهد بود، پس با توجه به این عبارت، اگر گرامری LR(1) نیست پس SLR(1) نیز نخواهد بود.
در نتیجه گرامر مورد نظر ما SLR(1) هم نیست.

قسمت د:

\begin{center}
    \begin{latin}
    0.$R' \rightarrow R$
    \\
    1.$R \rightarrow R|S$
    \\
    2.$R \rightarrow S$
    \\
    3.$S \rightarrow ST$
    \\
    4.$S \rightarrow T$
    \\
    5.$T \rightarrow {T}^{*}$
    \\
    6.$T \rightarrow U$
    \\
    7.$U \rightarrow a | b | c .... | z$
    \\
    8.$U \rightarrow (R)$
    \\
    \end{latin}
\end{center}

قسمت ه:

\begin{center}
\begin{latin}
\begin{tikzpicture}
    [node distance=1.5cm]
    \node[state] (R') {R'};

    %right
    \node[state,below of=R'] (R0) {R};
    \node[state,yshift=-1.5cm,right of=R0] (S0) {S};
    \node[state,yshift=-1.5cm,right of=S0] (R3) {S};
    \node[state,yshift=-1.5cm,right of=R3] (S4) {S};
    \node[state,yshift=-1.5cm,right of=S4] (T3) {T};
    \node[state,yshift=-1.5cm,right of=T3] (U2) {U};
    \node[state,below of=U2] (E) {e};
    
    %left of S0
    \node[state,left of=S0] (OR0) {|};
    \node[state,left of=OR0] (R1) {R};
    
    \node[state,below of=OR0] (S1) {S};
    \node[state,left of=S1] (OR1) {|};
    \node[state,left of=OR1] (R2) {R};

    %left
    \node[state,yshift=-1.5cm,left of=R2] (S2) {S};
    \node[state,yshift=-1.5cm,left of=S2] (T1) {T};
    \node[state,yshift=-1.5cm,left of=T1] (U1) {U};
    \node[state,below of=U1] (A) {a};

    \node[state,xshift=-1cm,below of=S1] (S3) {S};
    \node[state,xshift=1cm,below of=S1] (T0) {T};

    \node[state,xshift=1cm,below of=T0] (U0) {U};
    \node[state,xshift=1cm,below of=U0] (D) {d};

    \node[state,xshift=1cm,below of=S3] (T2) {T};
    \node[state,xshift=-1cm,below of=S3] (S5) {S};

    \node[state,xshift=-1cm,below of=S5] (T4) {T};
    \node[state,xshift=-1cm,below of=T4] (U3) {U};
    \node[state,xshift=-1cm,below of=U3] (B) {b};


    \node[state,xshift=-1cm,below of=T2] (T5) {T};
    \node[state,xshift=1cm,below of=T2] (ST) {*};

    \node[state,below of=T5] (U4) {U};
    \node[state,below of=U4] (C) {c};

    \draw[-] (R') to (R0);
    \draw[-] (R0) to (R1);
    \draw[-] (R0) to (OR0);
    \draw[-] (R0) to (S0);
    \draw[-] (R1) to (R2);
    \draw[-] (R1) to (OR1);
    \draw[-] (R1) to (S1);
    \draw[-] (S0) to (R3);
    \draw[-] (R2) to (S2);
    \draw[-] (S1) to (S3);
    \draw[-] (S1) to (T0);
    \draw[-] (R3) to (S4);
    \draw[-] (S2) to (T1);
    \draw[-] (S3) to (S5);
    \draw[-] (S3) to (T2);
    \draw[-] (T0) to (U0);
    \draw[-] (S4) to (T3);
    \draw[-] (T1) to (U1);
    \draw[-] (S5) to (T4);
    \draw[-] (T2) to (T5);
    \draw[-] (T2) to (ST);
    \draw[-] (U0) to (D);
    \draw[-] (T3) to (U2);
    \draw[-] (U1) to (A);
    \draw[-] (T4) to (U3);
    \draw[-] (U3) to (B);
    \draw[-] (T5) to (U4);
    \draw[-] (U4) to (C);
    \draw[-] (U1) to (A);
    \draw[-] (U2) to (E);



\end{tikzpicture}
\end{latin}
\end{center}
