%نام و نام خانوادگی:
%شماره دانشجویی: 
\مسئله{نام سؤال}

\پاسخ{}

ابتدا گرامر زیر را به گرارمر ها اضافه می‌کنیم تا عمل accept کردن درست صورت بگیرد.

\begin{center}
\begin{latin}
$S' \rightarrow S$
\end{latin}
\end{center}

پس گرامر ما به صورت زیر می‌شود:

\begin{center}
\begin{latin}
$S' \rightarrow S$
\\
$S \rightarrow (L) | a$
\\
$L \rightarrow L,S | S$
\end{latin}
\end{center}

سپس اقدام به ساخت جدول first و follow می‌کنیم.

\begin{center}
    \begin{latin}
        \begin{tabular}{|c|c|c|}
            \hline
            Non Terminal & first & follow \\ \hline
            S'           & ( a   & \$     \\ \hline
            S            & ( a   & , ) \$  \\ \hline
            P            & ( a   & , ) \$  \\ \hline
        \end{tabular}
    \end{latin}
\end{center}

سپس اقدام به ساخت DFA مربوط به LR(1) می‌کنیم.


\begin{latin}
\begin{tikzpicture}
    [>=latex, 
    node distance=2.5cm, 
    block/.style={state, rectangle, text width=8em}
    ]
    \node [block, label=above right: 0] (q0) 
    {
        \( S' \rightarrow . S \hfill \$\)
        \\
        \( S \rightarrow . (L) \hfill \$\)
        \\
        \( S \rightarrow . a \hfill \$\)
        \\
        \( L \rightarrow . L,S \hfill \$\)
        \\
        \( L \rightarrow . S \hfill \$\)
        \\
    };

    \node [block, label=above right: 1] (q1) [above of=q0]
    {
        \( S' \rightarrow S . \hfill \$\)
        \\
    };

    \node [block, label=above right: 3] (q3) [below of=q0]
    {
        \( S \rightarrow a . \hfill \$\)
        \\
    };


    \node [block, label=above right: 2] (q2)  [right=1cm of q0]
    {
        \( S \rightarrow (.L) \hfill \$\)
        \\
        \( L \rightarrow  .L,S \hfill ) \; ,\)
        \\
        \( L \rightarrow  .S \hfill ) \; ,\)
        \\
        \( S \rightarrow .(L) \hfill ) \; ,\)
        \\
        \( S \rightarrow .a \hfill ) \; ,\)
    };


    \node [block, label=above right: 4] (q4)  [above=1cm of q2]
    {
        \( S \rightarrow (L.) \hfill \$\)
        \\
        \( L \rightarrow  L.,S \hfill ) \; ,\)
        \\
    };


    \node [block, label=above right: 8] (q8)  [above of= q4]
    {
        \( S \rightarrow (L). \hfill \$\)
        \\
    };

    \node [block, label=above right: 9] (q9)  [right=1cm of q4]
    {
        \( L \rightarrow L,.S \hfill ) \; ,\)
        \\
        \( S \rightarrow .(L) \hfill ) \; ,\)
        \\
        \( S \rightarrow .a \hfill ) \; ,\)
        \\
    };


    \node [block, label=above right: 5] (q5)  [below of=q2]
    {
        \( L \rightarrow S. \hfill ) \; , \)
        \\
    };


    \node [block, label=above left: 7] (q7)  [below of=q5]
    {
        \( S \rightarrow a. \hfill ) \; , \)
        \\
    };

    \node [block, label=above right: 11] (q11)  [above of = q9]
    {
        \( L \rightarrow L,S. \hfill ) \; , \)
        \\
    };

    \node [block, label=above right: 6] (q6)  [right=1cm of q2]
    {
        \( S \rightarrow (.L) \hfill ) \; , \)
        \\
        \( L \rightarrow  .L,S \hfill ) \; ,\)
        \\
        \( L \rightarrow  .S \hfill ) \; ,\)
        \\
        \( S \rightarrow .(L) \hfill ) \; ,\)
        \\
        \( S \rightarrow .a \hfill ) \; ,\)

    };


    \node [block, label=above right: 10] (q10)  [below=1cm of q6] {
        \( S \rightarrow (L.) \hfill ) \; , \)
        \\
        \( L \rightarrow  L.,S \hfill ) \; , \)
    };


    \node [block, label=above right: 12] (q12)  [below of=q10]
    {
        \( L \rightarrow (L). \hfill ) \; , \)
        \\
    };

    \draw[->] (q0) edge[above] node[right] {S} (q1);
    \draw[->] (q0) edge[above] node[right] {a} (q3);
    \draw[->] (q0) edge[above] node[above] {(} (q2);
    \draw[->] (q2) edge[above] node[right] {L} (q4);
    \draw[->] (q2) edge[above] node[right] {S} (q5);
    \draw[->] (q2) edge[bend left,looseness=0.5,out=305,in=255] node[left] {a} (q7);
    \draw[->] (q2) edge[above] node[above] {(} (q6);
    \draw[->] (q4) edge[above] node[right] {)} (q8);
    \draw[->] (q4) edge[above] node[above] {,} (q9);
    \draw[->] (q6) edge[loop right,looseness=2] node[right] {(} (q6);
    \draw[->] (q6) edge[above] node[right] {a} (q7);
    \draw[->] (q6) edge[above] node[left] {L} (q10);
    \draw[->] (q9) edge[above] node[left] {(} (q6);
    \draw[->] (q9) edge[above] node[left] {S} (q11);
    \node [xshift=1.8cm] (foo) at (q12.east) {a};
    \draw[->] (q9.east) to  [out=0,in=90] ([xshift=0.2cm]foo.east) to [out=-90,in=-90](q7.south);

    \draw[->] (q10) edge[above] node[left] {)} (q12);
    \draw[->] (q10) edge[bend right,out=280,in=255] node[right] {,} (q9);

    \node [yshift=1cm] (acc) at (q1.north) {accept};
    \draw[->] (q1) edge[above] node[left] {\$} (acc);

\end{tikzpicture} 
\end{latin}

جدول پارس به صورت زیر است:


\begin{center}
\begin{latin}
\begin{tabular}{|c|c|c|c|c|c|c|c|}
\hline
State & (  & )                      & a  & ,                      & \$                     & S   & L   \\ \hline
0     & S2 &                        & S3 &                        &                        & G1  &     \\ \hline
1     &    &                        &    &                        & acc                    &     &     \\ \hline
2     & S6 &                        & S7 &                        &                        & G5  & G4  \\ \hline
3     &    &                        &    &                        & R(S$\rightarrow${}a)   &     &     \\ \hline
4     &    & S8                     &    & S9                     &                        &     &     \\ \hline
5     &    & R(L $\rightarrow$ S)   &    & R(L$\rightarrow${}S)   &                        &     &     \\ \hline
6     & S6 &                        & S7 &                        &                        & G5  & G10 \\ \hline
7     &    & R(S$\rightarrow${}a)   &    & R(S$\rightarrow${}a)   &                        &     &     \\ \hline
8     &    &                        &    &                        & R(S$\rightarrow${}(L)) &     &     \\ \hline
9     & S6 &                        & S7 &                        &                        & G11 &     \\ \hline
10    &    & S12                    &    & S9                     &                        &     &     \\ \hline
11    &    & R(L$\rightarrow${}L,S) &    & R(L$\rightarrow${}L,S) &                        &     &     \\ \hline
12    &    & R(S$\rightarrow${}(L)) &    & R(S$\rightarrow${}(L)) &                        &     &     \\ \hline
\end{tabular}
\end{latin}
\end{center}

و در نهایت مراحل پارس به صورت زیر است:

\begin{center}
\begin{latin}
\begin{tabular}{|c|c|c|c|}
\hline
Stack         & Symbol & Input         & Action \\ \hline
0             &        & ((a,a),a,a)\$ & shift  \\ \hline
0,2           & (      & (a,a),a,a)\$  & shift  \\ \hline
0,2,6         & ((     & a,a),a,a)\$   & shift  \\ \hline
0,2,6,7       & ((a    & ,a),a,a)\$    & reduce \\ \hline
0,2,6,5       & ((S    & ,a),a,a)\$    & reduce \\ \hline
0,2,6,10      & ((L    & ,a),a,a)\$    & shift  \\ \hline
0,2,6,10,9    & ((L,   & a),a,a)\$     & shift  \\ \hline
0,2,6,10,9,7  & ((L,a  & ),a,a)\$      & reduce \\ \hline
0,2,6,10,9,11 & ((L,S  & ),a,a)\$      & reduce \\ \hline
0,2,6,10      & ((L    & ),a,a)\$      & shift  \\ \hline
0,2,6,10,12   & ((L)   & ,a,a)\$       & reduce \\ \hline
0,2,5         & (S     & ,a,a)\$       & reduce \\ \hline
0,2,4         & (L     & ,a,a)\$       & shift  \\ \hline
0,2,4,9       & (L,    & a,a)\$        & shift  \\ \hline
0,2,4,9,7     & (L,a   & ,a)\$         & reduce \\ \hline
0,2,4,9,11    & (L,S   & ,a)\$         & reduce \\ \hline
0,2,4         & (L     & ,a)\$         & shift  \\ \hline
0,2,4,9       & (L,    & a)\$          & shift  \\ \hline
0,2,4,9,7     & (L,a   & )\$           & reduce \\ \hline
0,2,4,9,11    & (L,S   & )\$           & reduce \\ \hline
0,2,4         & (L     & )\$           & shift  \\ \hline
0,2,4,8       & (L)    & \$            & reduce \\ \hline
0,1           & S      & \$            & accept \\ \hline
\end{tabular}
\end{latin}
\end{center}
