%نام و نام خانوادگی:
%شماره دانشجویی: 
\مسئله{نام سؤال}

\پاسخ{}

اولی LR(0) است ولی دومی خیر.
\\
برای گرامر اول:
\\
ابتدا S' را تعریف می‌کنیم که accept و reduce با هم تلاقی نکنن.

\begin{latin}
\begin{center}
$S' \rightarrow S$
\\
$S \rightarrow S + E | E$
\\
$E \rightarrow number \; | \; (S)$
\\
\end{center}
\end{latin}


حال اقدام به ساخت آتاماتای LR(0) می‌کنیم.
\\

\begin{latin}
\begin{tikzpicture}
    [>=latex, 
    node distance=1cm, 
    block/.style={state, rectangle, text width=8em}
    ]
    \node [block, label=above right: 0] (q0) 
    {
        \( S' \rightarrow . S \)
        \\
        \( S \rightarrow . S+E\)
        \\
        \( S \rightarrow . E \)
        \\
        \( E \rightarrow . number \)
        \\
        \( E \rightarrow . (S) \)
        \\
    };

    \node [block, label=above right: 1, above=1cm of q0] (q1)
    {
        \(S' \rightarrow S.\)
        \\
        \(S \rightarrow S. + E \)
        \\
    };

    \node [block, label=above right: 2, right=1cm of q0] (q2)
    {
        \(S \rightarrow E.\)
        \\
    };

    \node [block, label=above right: 3, below=1cm of q0] (q3)
    {
        \(E \rightarrow \; number. \)

    };

    \node [block, label=above right: 4, right=1cm of q3] (q4)
    {
        \(S \rightarrow (.S) \)
        \\
        \(S \rightarrow .S + E \)
        \\
        \(S \rightarrow .E \)
        \\
        \(E \rightarrow .number \)
        \\
        \(E \rightarrow .(S) \)
    };

    \node [block, label=above right: 5, right=1cm of q2] (q5)
    {
        \(S \rightarrow S + . E  \)
        \\
        \(E \rightarrow .number \)
        \\
        \(E \rightarrow .(S) \)
    };


    \node [block, label=above right: 6, right=1cm of q4] (q6)
    {
        \(S \rightarrow (S.) \)
        \\
        \(S \rightarrow S.+ E \)
        \\
    };


    \node [block, label=above right: 7, above=1cm of q5] (q7)
    {
        \(S \rightarrow S + E.  \)
        \\
    };

    \node [block, label=above right: 8, below=1cm of q6] (q8)
    {
        \(S \rightarrow (S).  \)
        \\
    };

    
    \draw[->] (q0) edge[above] node[right] {S} (q1);
    \draw[->] (q0) edge[above] node[above] {E} (q2);
    \draw[->] (q0) edge[above] node[left] {number} (q3);
    \draw[->] (q0) edge[above] node[above] {(} (q4);
    \draw[->] (q1) edge[above] node[above] {+} (q5);
    \draw[->] (q4) edge[above] node[above] {S} (q6);
    \draw[->] (q4.south west) edge[above] node[below] {number} (q3.south);
    \draw[->] (q5) edge[above] node[right] {(} (q4);
    \draw[->] (q5) edge[above] node[right] {E} (q7);
    \draw[->] (q6) edge[above] node[right] {)} (q8);
    \draw[->] (q6) edge[above] node[right] {+} (q5);

    \node [yshift=1cm] (acc) at (q1.north) {accept};
    \draw[->] (q1) edge[above] node[left] {\$} (acc);

\end{tikzpicture} 
\end{latin}

کانفلیکتی وجود ندارد پس این یک گرامر LR(0) است.
\\ 
اما گرامر دوم یک گرامر LR(0) نیست. چرا که از وضعیت شروع با خواندن E به وضعیت زیر می‌رسیم.

\begin{latin}
\begin{tikzpicture}
    [>=latex, 
    node distance=1cm, 
    block/.style={state, rectangle, text width=8em}
    ]

    \node [block, label=above right: 0] (q0) 
    {
        \( S' \rightarrow . S \)
        \\
        \( S \rightarrow . E+S\)
        \\
        \( S \rightarrow . E \)
        \\
        \( E \rightarrow . number \)
        \\
        \( E \rightarrow . (S) \)
        \\
    };


    \node [block, label=above right: 1, right=1cm of q0] (q1) 
    {
        \( S \rightarrow E. + S\)
        \\
        \( S \rightarrow E. \)
        \\
    };

    \draw[->] (q0) edge[above] node[above] {E} (q1);
 
\end{tikzpicture} 
\end{latin}

پس ما یک \lr{shift/reduce conflict} داریم و در نتیجه این گرامر LR(0) نیست.
