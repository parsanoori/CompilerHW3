%نام و نام خانوادگی:
%شماره دانشجویی: 
\مسئله{نام سؤال}

\پاسخ{}

الف)

به مجموعه قوانین یک D' اضافه می‌کنیم تا اینکه عمل accept ممکن باشد. پس داریم:

\begin{center}
\begin{latin}
$D' \rightarrow D$
\\
$D \rightarrow T | S$
\\
$S \rightarrow T \; like \;A$
\\
$T \rightarrow beautifull \; | \; small$
\\
$A \rightarrow rabbit \;| \;flower$
\end{latin}
\end{center}

سپس اقدام به ساخت DFA مربروط به LR(0) می‌کنیم.

\begin{latin}
\begin{tikzpicture}
    [>=latex, 
    node distance=3cm, 
    block/.style={state, rectangle, text width=8em}
    ]

    \node [block, label=above right: 0] (q0)
    {
        \(D' \rightarrow .D\)
        \\
        \(D \rightarrow .T\)
        \\
        \(D \rightarrow .S\)
        \\
        \(T \rightarrow .\; beautifull\)
        \\
        \(T \rightarrow .\; small\)
        \\
        \(S \rightarrow .\; T \; like \; A\)
    };

    \node [block, label=above right: 1] (q1) [above of=q0]
    {
        \( D' \rightarrow D . \hfill \$\)
        \\
    };
    

    \node [block, label=above right: 3] (q3) [right=1cm of q0]
    {
        \(D \rightarrow S.\)
        \\
    };

    \node [block, label=above right: 2] (q2) [above of=q3]
    {
        \(D \rightarrow T.\)
        \\
        \(S \rightarrow T. \; like \; A\)
        \\
    };

    \node [block, label=above right: 4] (q4) [below of=q3]
    {
        \(T \rightarrow \; beautifull.\)
        \\
    };

    \node [block, label=above right: 5] (q5) [below of=q0]
    {
        \(T \rightarrow \; small.\)
        \\
    };
 
    \node [block, label=above right: 6] (q6) [right=1cm of q2]
    {
        \(S \rightarrow \; T \; like . \; A\)
        \\
        \(A \rightarrow \; . rabbit  \)
        \\
        \(A \rightarrow \; . flower  \)
        \\
    };


    \node [block, label=above right: 7] (q7) [right=1cm of q6]
    {
        \(S \rightarrow \; T \; like  \; A.\)
        \\
    };

    \node [block, label=above right: 8] (q8) [below of=q7]
    {
        \(A \rightarrow \; rabbit.\)
        \\
    };


    \node [block, label=above right: 9] (q9) [below of=q8]
    {
        \(A \rightarrow \; flower.\)
        \\
    };

    \draw[->] (q0) edge[above] node[right] {D} (q1);
    \draw[->] (q0) edge[above] node[left] {small} (q5);
    \draw[->] (q0) edge[above] node[left] {T} (q2);
    \draw[->] (q0) edge[above] node[above] {S} (q3);
    \draw[->] (q0) edge[above] node[right] {beautifull} (q4);
    \draw[->] (q2) edge[above] node[above] {like} (q6);
    \draw[->] (q6) edge[above] node[above] {A} (q7);
    \draw[->] (q6) edge[above] node[right] {rabbit} (q8);
    \draw[->] (q6) edge[above] node[left] {flower} (q9.west);

    \node [yshift=1cm] (acc) at (q1.north) {accept};
    \draw[->] (q1) edge[above] node[left] {\$} (acc);

\end{tikzpicture}
\end{latin}

جدول پارس LR(0) به صو.رت زیر است(به دلیل نبود فضا دو قسمت شد):

\begin{center}
\begin{latin}
\begin{tabular}{|l|l|l|l|l|}
\hline
State & beautiful                    & flower                       & like                         & rabbit                       \\ \hline
0     & S4                           &                              &                              &                              \\ \hline
1     & accept                       & accept                       & \textbf{accept}              & accept                       \\ \hline
2     & R(D-\textgreater{}T)         & R(D-\textgreater{}T)         & S6 / R(D-\textgreater{}T)    & R(D-\textgreater{}T)         \\ \hline
3     & R(D-\textgreater{}S)         & R(D-\textgreater{}S)         & R(D-\textgreater{}S)         & R(D-\textgreater{}S)         \\ \hline
4     & R(T-\textgreater{}beuatiful) & R(T-\textgreater{}beuatiful) & R(T-\textgreater{}beuatiful) & R(T-\textgreater{}beuatiful) \\ \hline
5     & R(T-\textgreater{}small)     & R(T-\textgreater{}small)     & R(T-\textgreater{}small)     & R(T-\textgreater{}small)     \\ \hline
6     &                              & S9                           &                              & S8                           \\ \hline
7     & R(S-\textgreater{}T like A)  & R(S-\textgreater{}T like A)  & R(S-\textgreater{}T like A)  & R(S-\textgreater{}T like A)  \\ \hline
8     & R(A-\textgreater{}rabbit)    & R(A-\textgreater{}rabbit)    & R(A-\textgreater{}rabbit)    & R(A-\textgreater{}rabbit)    \\ \hline
9     & R(A-\textgreater{}flower)    & R(A-\textgreater{}flower)    & R(A-\textgreater{}flower)    & R(A-\textgreater{}flower)    \\ \hline
\end{tabular}

\hspace*{-1cm}
\begin{tabular}{|l|l|l|l|l|l|}
\hline
State & small                        & D                            & S                            & T                            & A                            \\ \hline
0     & S5                           & G1                           & G3                           & G2                           &                              \\ \hline
1     & accept                       & accept                       & accept                       & accept                       & accept                       \\ \hline
2     & R(D-\textgreater{}T)         & R(D-\textgreater{}T)         & R(D-\textgreater{}T)         & R(D-\textgreater{}T)         & R(D-\textgreater{}T)         \\ \hline
3     & R(D-\textgreater{}S)         & R(D-\textgreater{}S)         & R(D-\textgreater{}S)         & R(D-\textgreater{}S)         & R(D-\textgreater{}S)         \\ \hline
4     & R(T-\textgreater{}beuatiful) & R(T-\textgreater{}beuatiful) & R(T-\textgreater{}beuatiful) & R(T-\textgreater{}beuatiful) & R(T-\textgreater{}beuatiful) \\ \hline
5     & R(T-\textgreater{}small)     & R(T-\textgreater{}small)     & R(T-\textgreater{}small)     & R(T-\textgreater{}small)     & R(T-\textgreater{}small)     \\ \hline
6     &                              &                              &                              &                              & G7                           \\ \hline
7     & R(S-\textgreater{}T like A)  & R(S-\textgreater{}T like A)  & R(S-\textgreater{}T like A)  & R(S-\textgreater{}T like A)  & R(S-\textgreater{}T like A)  \\ \hline
8     & R(A-\textgreater{}rabbit)    & R(A-\textgreater{}rabbit)    & R(A-\textgreater{}rabbit)    & R(A-\textgreater{}rabbit)    & R(A-\textgreater{}rabbit)    \\ \hline
9     & R(A-\textgreater{}flower)    & R(A-\textgreater{}flower)    & R(A-\textgreater{}flower)    & R(A-\textgreater{}flower)    & R(A-\textgreater{}flower)    \\ \hline
\end{tabular}

\end{latin}
\end{center}


ب)


\begin{center}
\begin{latin}
\begin{tabular}{|l|l|l|l|}
\hline
State & first            & follow  & nullable \\ \hline
D'    & beeatiful, small & \$      & false    \\ \hline
D     & beeatiful, small & \$      & false    \\ \hline
S     & beeatiful, small & \$      & fasle    \\ \hline
T     & beeatiful, small & like \$ & false    \\ \hline
A     & rabbit, flower   & \$      & false    \\ \hline
\end{tabular}
\end{latin}
\end{center}

ج)

جدول پارس SLR به صورت زیر است:
\begin{latin}
\begin{center}
\hspace*{-0.5cm}
\begin{tabular}{|c|c|c|c|c|c|c|c|c|c|c|}
\hline
State & beautiful & flower & like                         & rabbit & small & \$                           & D & S & T & A \\ \hline
0     & S4        &        &                              &        & S5    &                              & 1 & 3 & 2 &   \\ \hline
1     &           &        &                              &        &       & accept                       &   &   &   &   \\ \hline
2     &           &        & S6                           &        &       & r(D-\textgreater{}T)         &   &   &   &   \\ \hline
3     &           &        &                             &        &       & r(D-\textgreater{}S)         &   &   &   &   \\ \hline
4     &           &        & R(T-\textgreater{}beuatiful) &        &       & R(T-\textgreater{}beautiful) &   &   &   &   \\ \hline
5     &           &        & R(T-\textgreater{}small)     &        &       & R(T-\textgreater{}small)     &   &   &   &   \\ \hline
6     &           & S9     &                              & S8     &       &                              &   &   &   & 7 \\ \hline
7     &           &        &                              &        &       & R(S-\textgreater{}T like A)  &   &   &   &   \\ \hline
8     &           &        &                              &        &       & R(A-\textgreater rabbit)     &   &   &   &   \\ \hline
9     &           &        &                              &        &       & R(A -\textgreater flower)    &   &   &   &   \\ \hline
\end{tabular}
\end{center}
\end{latin}


د)

LR(0) نیست چرا که در state مشاره  2 نمی‌دانیم reduce کنیم یا shift ولی SLR هست چرا که با دانستن توکن بعدی این موضوع رفع می‌شود و می‌بینیم که جدول SLR  بدون مشکل است.
