%نام و نام خانوادگی:
%شماره دانشجویی: 
\مسئله{نام سؤال}

\پاسخ{}

الف)

ابتدا برای راحتی کار و اختصار , گرامر را به شکل زیر بازنویسی میکنیم:


\begin{center}
\begin{latin}
$S \rightarrow Pro$
\\
$Pro \rightarrow Tp \; t\_id \; (PL);$
\\
$Tp \rightarrow t\_i \; | \; t\_d \; | \;  t\_c$
\\
$PL \rightarrow PL,P \;  | \; P$
\\
$P \rightarrow Tp \; t\_id$
\end{latin}
\end{center}

سپس قوانین آن را جداگانه مینویسیم:

\begin{flushleft}
\begin{latin}
$1 -  S \rightarrow Pro$
\\
$2 -  Pro \rightarrow Tp \; t\_id \; (PL);$
\\
$3 -  Tp \rightarrow t\_i $
\\
$4 -  Tp \rightarrow t\_d $
\\
$5 -  Tp \rightarrow t\_c $
\\
$6 -  PL \rightarrow PL,P$
\\
$7 -  PL \rightarrow P$
\\
$8 -  P \rightarrow Tp \; t\_id$
\\
\end{latin}
\end{flushleft}

اکنون میتوانیم نمودار LR مربروط به این گرامر را رسم کنیم.

\begin{latin}
\begin{tikzpicture}
    [>=latex, 
    node distance=3.5cm, 
    block/.style={state, rectangle, text width=10em}
    ]

    \node [block, label=above right: 0] (q0)
    {
        \(S \rightarrow .Pro\)
        \\
        \(Pro \rightarrow .Tp \; t\_id \; (PL);\)
        \\
        \(Tp \rightarrow .t\_i\)
        \\
        \(Tp \rightarrow .t\_d\)
        \\
        \(Tp \rightarrow .t\_c\)
        \\
    };

    \node [block, label=above right: 1] (q1) [above=2cm of q0]
    {
        \( S \rightarrow Pro . \hfill \$\)
        \\
    };
    
    \node [yshift=1cm] (acc) at (q1.north) {accept};

    \node [block, label=above right: 2] (q2) [above right=2cm of q1]
    {
        \(Pro \rightarrow Tp.t\_id \; (PL);\)
        \\
    };

    \node [block, label=above right: 3] (q3) [right=1cm of q2]
    {
        \(Pro \rightarrow Tp \; t\_id . (PL);\)
        \\
    };
    
    \node [block, label=above right: 4] (q4) [below=1cm of q3]
    {
        \(Pro \rightarrow Tp \; t\_id \; ( . PL);\)
        \\
        \(PL \rightarrow .PL,P \)
        \\
        \(PL \rightarrow .P \)
        \\
        \(P \rightarrow .Tp \; t\_i\)
        \\
        \(Tp \rightarrow .t\_i\)
        \\
        \(Tp \rightarrow .t\_d\)
        \\
        \(Tp \rightarrow .t\_c\)
        \\
    };
    
    \node [block, label=above right: 5] (q5) [below=1cm of q2]
    {
        \(Pro \rightarrow Tp \; t\_id \; (PL.);\)
        \\
        \(PL \rightarrow  PL.,P \)
        \\
    };
    
    \node [block, label=above right: 6] (q6) [below=1cm of q5]
    {
        \(Pro \rightarrow Tp \; t\_id \; (PL).;\)
        \\
    };
    
    \node [block, label=above right: 7] (q7) [below=1cm of q6]
    {
        \(Pro \rightarrow Tp \; t\_id \; (PL);.;\)
        \\
    };

    \node [block, label=above right: 8] (q8) [below=1cm of q7]
    {
        \(PL \rightarrow PL,.P \)
        \\
        \(P \rightarrow .Tp \; t\_id\)
        \\
        \(Tp \rightarrow .t\_i\)
        \\
        \(Tp \rightarrow .t\_d\)
        \\
        \(Tp \rightarrow .t\_c\)
        \\
    };
    
    \node [block, label=above right: 9] (q9) [below=1cm of q8]
    {
        \(PL \rightarrow PL,P. \)
        \\
    };
    
    \node [block, label=above right: 10] (q10) [below=1cm of q9]
    {
        \(P \rightarrow Tp \;. t\_id  \)
        \\
    };
    
    \node [block, label=above right: 11] (q11) [below=1cm of q10]
    {
        \(P \rightarrow Tp \; t\_id . \)
        \\
    };
    
    \node [block, label=above right: 12] (q12) [below=1cm of q11]
    {
        \(Tp \rightarrow t\_i . \)
        \\
    };
    
    \node [block, label=above right: 13] (q13) [below=1cm of q12]
    {
        \(Tp \rightarrow t\_d . \)
        \\
    };
    
    \node [block, label=above right: 14] (q14) [below=1cm of q13]
    {
        \(Tp \rightarrow t\_c . \)
        \\
    };
    
    \node [block, label=above right: 15] (q15) [below=17cm of q4]
    {
        \(PL \rightarrow P . \)
        \\
    };
    
    \draw[->] (q0) edge[above] node[left] {Pro} (q1);
    \draw[->] (q1) edge[above] node[left] {\$} (acc);
    \draw[->] (q0) edge[bend left,looseness=0.5,out=-30,in=120] node[above=2.5] {Tp} (q2);
    \draw[->] (q2) edge[right] node[above] {t\_id} (q3);
    \draw[->] (q3) edge[below] node[left] {(} (q4);
    \draw[->] (q4) edge[left] node[above] {PL} (q5);
    \draw[->] (q5) edge[below] node[left] {)} (q6);
    \draw[->] (q6) edge[right] node[left] {;} (q7);
    \draw[->] (q5) edge[bend right,looseness=0.20,out=273,in=267] node[left] {,} (q8);
    \draw[->] (q8) edge[below] node[left] {P} (q9);
    \draw[->] (q8) edge[below] node[left] {P} (q9);
    \draw[->] (q8) edge[bend right,looseness=0.25,out=90,in=90] node[right] {,} (q10);
    \draw[->] (q10) edge[below] node[left] {t\_id} (q11);    
    \draw[->] (q8) edge[bend right,looseness=0.15,out=273,in=267] node[above] {t\_i} (q12);
    \draw[->] (q8) edge[bend right,looseness=0.35,out=273,in=267] node[above] {t\_d} (q13);    
    \draw[->] (q8) edge[bend right,looseness=0.5,out=273,in=267] node[above] {t\_c} (q14);    
    \draw[->] (q0) edge[bend right,looseness=0.5,out=300,in=245] node[above] {t\_i} (q12);
    \draw[->] (q0) edge[bend right,looseness=0.5,out=300,in=245] node[above] {t\_d} (q13);    
    \draw[->] (q0) edge[bend right,looseness=0.5,out=300,in=245] node[above] {t\_c} (q14);
    \draw[->] (q4) edge[bend left,looseness=0.75,out=0,in=110] node[below=1] {t\_i} (q12);
    \draw[->] (q4) edge[bend left,looseness=0.75,out=0,in=110] node[below=2] {t\_d} (q13);    
    \draw[->] (q4) edge[bend left,looseness=0.75,out=0,in=110] node[below=4] {t\_c} (q14);
    \draw[->] (q4) edge[bend left,looseness=0.35,out=20,in=170] node[right] {P} (q15);

\end{tikzpicture}
\end{latin}


با توجه به این که دیده میشود در برخی موارد با یک مقدار از چند وضعیت به یک وضعیت یکسان میرویم , شاید گمان کنید که با خطای reduce/reduce conflict ممکن است روبرو شویم , اما همانطور که در قسمت ج خواهیم گفت , چنین ایرادی به وجود نخواهد آمد.

همچنین مشکل دیگری که باعث میشود گرامر مورد نظر LR(0) نشود آن است که در یک وضعیت , دو قانون داشته باشیم که:

 ۱ - در یکی از آن ها نقطه در انتهای سمت راست باشد, که معنایش این است که باید پذیرش انجام شود. 
 
 ۲ - و در دیگری نقطه در میانه های سمت راست قانون باشد, که معنایش این است که باید عمل shift انجام داده و به مسیر ادامه دهیم.
 
خوشبختانه در این نمودار این حالت وجود ندارد و بنابراین خطای shift/reduce conflict هم نخواهیم داشت. پس میتوان گفت این گرامر , LR(0) است.

ب)

ابتدا جدول پارس را برای این گرامر رسم میکنیم:

\begin{center}
\begin{latin}
\begin{tabular}{|c|c|c|c|c|c|c|c|}
\hline
State & (   & )                      & ,   & ;                      & t\_i                   & t\_d & t\_c \\ \hline
0     &     &                        &     &                        & S12                    & S13  & S14  \\ \hline
1     & ACC & ACC                    & ACC & ACC                    & ACC                    & ACC  & ACC  \\ \hline
2     &     &                        &     &                        & S3                     &      &      \\ \hline
3     & S4  &                        &     &                        &                        &      &      \\ \hline
4     &     & S8                     &     &                        & S12                    & S13  & S14  \\ \hline
5     &     & S6                     & S8  &                        &                        &      &      \\ \hline
6     &     &                        &     & S7                     &                        &      &      \\ \hline
7     & R2  & R2                     & R2  & R2                     & R2                     & R2   & R2   \\ \hline
8     &     &                        & S10 &                        & S12                    & S13  & S14  \\ \hline
9     & R6  & R6                     & R6  & R6                     & R6                     & R6   & R6   \\ \hline
10    &     &                        &     &                        &                        &      &      \\ \hline
11    & R8  & R8                     & R8  & R8                     & R8                     & R8   & R8   \\ \hline
12    & R3  & R3                     & R3  & R3                     & R3                     & R3   & R3   \\ \hline
11    & R4  & R4                     & R4  & R4                     & R4                     & R4   & R4   \\ \hline
11    & R5  & R5                     & R5  & R5                     & R5                     & R5   & R5   \\ \hline
11    & R7  & R7                     & R7  & R7                     & R7                     & R7   & R7   \\ \hline

\end{tabular}
\end{latin}
\end{center}

ادامه جدول پارس را در اینجا مشاهده میکنید:

\begin{center}
\begin{latin}
\begin{tabular}{|c|c|c|c|c|c|c|}
\hline
State & t\_id  & S                           & Pro                    & Tp                     & PL   & P   \\ \hline
0     &        &                             & G1                     & G2                     &      &     \\ \hline
1     & ACC    & ACC                         & ACC                    & ACC                    & ACC  & ACC \\ \hline
2     &        &                             &                        &                        &      &     \\ \hline
3     &        &                             &                        &                        &      &     \\ \hline
4     &        &                             &                        &                        & G5   & G15 \\ \hline
5     &        &                             &                        &                        &      &     \\ \hline
6     & S6     &                             &                        &                        &      &     \\ \hline
7     & R2     & R2                          & R2                     & R2                     & R2   & R2  \\ \hline
8     &        &                             &                        &                        &      &     \\ \hline
9     & R6     & R6                          & R6                     & R6                     & R6   & R6  \\ \hline
10    & S11    &                             &                        &                        &      &     \\ \hline
11    & R8     & R8                          & R8                     & R8                     & R8   & R8  \\ \hline
12    & R3     & R3                          & R3                     & R3                     & R3   & R3  \\ \hline
13    & R4     & R4                          & R4                     & R4                     & R4   & R4  \\ \hline
14    & R5     & R5                          & R5                     & R5                     & R5   & R5  \\ \hline
15    & R7     & R7                          & R7                     & R7                     & R7   & R7  \\ \hline

\end{tabular}
\end{latin}
\end{center}







ج)


